\documentclass{lni}
\let\ifpdf\relax
\IfFileExists{latin1.sty}{\usepackage{latin1}}{\usepackage{isolatin1}}

\usepackage{graphicx}

%neue Rechtschreibung
\usepackage{ngerman}
\usepackage[colorlinks=false,pdfborder=0 0 0]{hyperref} %URLS
\usepackage[hyphenbreaks]{breakurl}
\usepackage{listings} % XML Listing

\usepackage{acronym}
\usepackage{picins} % Grafiken mit hpic einfuegen
% deutsche Silbentrennung
\usepackage[ngerman]{babel} 
% wegen deutschen Umlauten
\usepackage[ansinew]{inputenc}

\author{
	Tobias Schmid, Andreas Kn�pfle \\ 
	\\ 
	Fachbereich Informatik \\ 
	Freie Universitt Berlin \\ 
	Anschrift \\ 
	Postleitzahl Ort \\ 
	emaiaddresse@autor1 \\
	emaiaddresse@autor2
}
\title{Communities im Co-Author-Graphen}
\begin{document}
\maketitle

\begin{abstract}
Im Rahmen einer Semersterarbeit werden die Metadaten des DBLP-Datensatzes analysiert. Diese Metadaten enthalten Informationen �ber die in der DBLP-Datenbank gespeicherten wissenschaftlichen Publikationen. Auf dieser Grundlage wird ein Co-Author-Graph erzeugt, der die im Datensatz auftretenden Autoren �ber ihre gemeinsamen Publikationen verbindet. Es wird analysiert, ob Gemeinschaften (Communities) unter den Autoren ausgemacht werden k�nnen und welche Rollen die Autoren innerhalb ihrer Gemeinschaften �bernehmen. Abschlie�end wird nach einer geeigneten M�glichkeit gesucht den Co-Author-Graphen und die Ergebnisse der Analyse auf geeignete Weise zu visualisieren.  
\end{abstract}
\newpage
\tableofcontents
\newpage
\section{Einleitung}

Der \acs{DBLP} Datensatz beinhaltet eine gro�e Sammlung wissenschaftlicher Publikationen und deren Meta-Daten aus dem Bereich Informatik. Insgesamt sind bereits mehr als 1,8 Millionen Publikationen aufgenommen worden (\cite{dblp_org}). Aus diesen Daten l�sst sich ein Co-Autor-Graph ableiten, welcher alle Autoren als Knoten und Publikationen zwischen ihnen als gemeinsame Kanten beschreibt. Dieser Graph kann durch graphentheoretische Betrachtungen analysiert werden.\\ 

Anschaulich hei�t das: Haben zwei Autoren miteinander publiziert, sind deren Knoten durch eine Kante verbunden. Das Gewicht einer Kante zeigt zus�tzlich die Anzahl der gemeinsamen Publikationen an.\\

Bei einem Datensatz dieser Gr��e k�nnte eine Analyse dieses Graphen repr�sentative Aussagen �ber die Autoren und deren Umfeld geben. 
Somit wird die wissenschaftliche Fragestellung dieser Arbeit wie folgt festgelegt: \\\\

 Welche Charaktereigenschaften weisen die Autoren und ihr Umfeld im Co-Autorgraph der DBLP auf? \\\\

% { Methoden }
Um dieser Frage nachzukommen, wird eine Infrastruktur entwickelt, die es erm�glicht den Co-Autor-Graphen zu generieren und zu visualisieren. Zus�tzlich m�ssen Werkzeuge entwickelt werden, die es erlauben den Graphen zu analysieren. 

% {Motivation}
% {Hauptfrage: Welche Charaktereigenschaften weisen die Autoren und ihr Umfeld im Co-Autorgraph der DBLP auf?}


\section{Ausgangssituation}

Als Datenquelle f�r den Co-Autor-Graphen werden die Metadaten der \acs{DBLP}-Datenbank im \acs{XML}-Format benutzt \cite{dblp_xml}.
Die Metadaten bestehen aus einer Liste der wissenschaftlichen Artikel mit Titel, Datum, Autoren und zus�tzlichen Informationen �ber die Ver�ffentlichung. \\

In \ref{listing:xml_source} ist ein Ausschnitt aus dieser \acs{XML} Quelle abgebildet. Aus dieser Quelle kann der Co-Autor-Graph erzeugt werden, da jeder Artikel alle beteiligten Autoren enth�lt und somit auf Knoten und Kanten im Graphen abgebildet werden kann.


%\subsubsection{Die DBLP Daten}


\section{Randbedingungen}
\label{randbendingingen}

Um den Co-Autor-Graphen zu analysieren, wird eine effiziente Form des Datenzugriffs ben�tigt. Um diesen Zugriff technisch umzusetzen wird im Weiteren die Graphdatenbank Neo4J verwendet. Mit Hilfe von Neo4J kann der Graph einfach und effizient gespeichert und traversiert werden. Zus�tzlich k�nnen Meta-Informationen zu den Knoten und Kanten in der Datenbank verwaltet werden, womit diese durch Analyseergebnisse angereichert werden k�nnen. Weiterhin unterst�tzt Neo4J die Entwicklung von Algorithmen auf Graphen und bietet bereits einige Implementierungen dieser.\\
Auf dieser Grundlage wird eine Applikation entwickelt, die sich aus vielen verschiedenen selbst�ndigen Modulen zusammensetzt. Alle Module sind nur �ber die Datenbasis Neo4J lose gekoppelt und k�nnen jeweils durch andere Module ersetzt werden. \\
Zum Zeitpunkt des Schreibens dieses Artikels sind folgende Module implementiert:
\begin{description}
  \item[parser] - Parsen der DBLP XML-Datei und Erzeugung des Co-Autor-Graphen und der Kantenliste f�r den Community-Algorithmus
  \item[importer] - Importieren der Ergebnisse des Community-Algorithmus
  \item[count] - Analyse der Gr��e der Communities
  \item[conductance] - Analyse der Conductance der Communities
  \item[roles] - Analyse der Rollen die die Autoren in ihrer Community �bernehmen
  \item[web] - Visualisieren der entstandenen Co-Autor-Graphen
\end{description}
Die Entwicklung der Applikation ist �ffentlich einsehbar\footnote{\burl{https://github.com/andreasknoepfle/dblp_communities}}. 


%\subsubsection{Neo4J Datenbank}

%\subsubsection{Neo4J Datenbank}


\section{Communities im Co-Autor-Graphen}


Die Autoren sind durch gemeinsame Publikationen im Co-Autor-Graphen miteinander verbunden. Viele Publikationen zwischen zwei Autoren lassen auf eine starke Verbindung zwischen diesen schlie�en. Durch diese Verbindungen k�nnen sich Gemeinschaften unter den Autoren bilden.\\

% {Teilfrage: Lassen sich Gemeinschaften/Communities in den dblp Daten finden?}
Im Folgenden wird also untersucht, ob im Co-Autor-Graphen der DBLP Gemeinschaften unter den Autoren zu finden sind. \\
Sollten Gemeinschaften gefunden werden k�nnen, ergeben sich die weiteren Fragestellungen:

	\item Wie stark unterscheiden sich die Communities in ihrer Gr��e ?
	\item Lassen sich die Communities charakterisieren ?
	\item Ist ein Zusammenhang zwischen Gr��e und Charakter einer Community erkennbar ?


% { Methoden }
%Community-Algorithmus
Um diese Gemeinschaften, weiterhin mit dem englischen Begriff Communities bezeichnet, finden zu k�nnen, wird der in \cite{communityDetection} beschriebene Community-Detection Algorithmus eingesetzt. Dieser Algorithmus findet Communities und deren Zusammensetzung als Hierarchiestruktur. Die Ergebnisse des Community-Detection Algorithmus werden in der Neo4J Datenbank gespeichert. \\

Die oberste ebene der Hierarchiestruktur, die die Zusammensetzung der Communities beschreibt, besteht aus den gefundenen Communities (Top-Level-Communities). Jede dieser Communities besteht wieder aus kleineren Communities und diese wieder aus Communities usw.. So gibt es mehrere Stufen, wobei die unterste Ebene die Autoren darstellt. \\

%Gr��e der Community
Die Gr��e jeder Top-Level-Community und ihrer Untergruppen kann durch einen rekursiven Algorithmus berechnet werden. \\

%Conductance einer Community
Die Conductance einer Community zeigt das Verh�ltnis der Anzahl der Kanten, welche diese Community mit einer anderen verbindet, zu der Anzahl der Kanten innerhalb der Community an. \\

Mit diesen beiden Werten k�nnen die Communities charakterisiert werden.

% { Ergebnisse }




% { Diskussion }

\section{Autoren}

Durch die im Abschnitt \ref{communities} gefundenen Communities kann man jetzt auch die Beziehung 
eines Autors zu der ihm zugeordneten Community betrachten. \\

In diesem Abschnitt wird untersucht ob verschiedene Rollenverteilungen unter den Autoren einer Community erkennbar sind.
Hierbei wird festgestellt welche Rollen unter den Autoren auftreten und welche Rolle etwa, die Autoren einnehmen die am 
meisten publizieren. \\

Dazu wurden die in \cite{hubsconnectors} beschriebene Klassifizierung von Knoten verwendet. Um die Knoten klassifizieren zu k�nnen 
muss jeweils die With-in-module-degree (z-score) und
f�r sie und ihre Communities berechnet werden. 
\begin{center}
					\hpic{\includegraphics[width=0.7\textwidth]{images/r1_r7.png}
					} \newcaption{Klassifizierung der Author-Knoten}
					\label{fig:r1_r7}
\end{center}
Anschlie�end 
kann die Klassifizierung des Knotens wie in Abbildung \ref{fig:r1_r7} beschrieben erfolgen. Grunds�tzlich wird in diesem Model 
zwischen folgenden Rollen unterschieden:

\begin{itemize}

\item Nabenknoten (hubs) 

\begin{description}
 \item [(R5) provincial hubs] (provinzielle Naben/Zentren)
 \item [(R6) connector hubs] (verbindende Naben/Zentren)
 \item [(R7) kinless hubs] (sippenlose Naben/Zentren)
\end{description}

 \item Einzelknoten (non-hubs) 
\begin{description}
 \item [(R1) ultraperipheral] (sehr dezentral)
 \item [(R2) peripheral] (dezentral)
 \item [(R3) connectors] (Bindeglieder)
 \item [(R4) kinless vertices] (Einzelknoten)
\end{description}

\end{itemize}

Zur Klassifizierung der Autor-Knoten im Co-Autor-Graphen, wird f�r jeden Knoten die Community betrachtet in der er sich befindet (Top-Level-Community). 


% {Teilfrage: Sind verschieden Rollenverteilungen unter den Autoren einer Community erkennbar ?}
% { Methoden }

% { Ergebnisse }
\begin{center}
					\hpic{\includegraphics[width=0.7\textwidth]{results/roles_hub_con}
					} \newcaption{Verteilung der Rollen hubs/connectors}
					\label{fig:hub_con_dist}
\end{center}

\begin{center}
					\hpic{\includegraphics[height=0.7\textwidth,angle=90]{results/roles_r1r7}
					} \newcaption{Verteilung der Rollen R1-R7}
					\label{fig:r1_r7_dist}
\end{center}
% { Diskussion }
\section{Visualisierung des Co-Autor-Graphen}
Um Informationen �ber einzelne Autoren und deren Communities aus dem Co-Autor-Graphen gewinnen zu k�nnen muss der Co-Author-Graph auf geeignete Art visualisiert werden. Daher wird versucht folgende Frage zu beantworten: 

 \begin{center}
\textit{L�sst sich das Umfeld eines Autors im Co-Autor-Graphen und seine Community visualisieren?}
\end{center}
 Der in der Neo4j-Datenbank gespeicherte Graph kann zwar �ber die in Neo4j eingebauten Bordmittel bereits teilweise visualisiert werden, allerdings funktioniert die Visualisierung bei dieser Datenmenge nur stark eingeschr�nkt. Vor allem die Tatsache, dass bei dieser Art von Visualisierung der jeweils betrachtete Teil des Graphen dynamisch aus der Datenbank gelesen werden muss, schr�nkt den Gesamtprozess stark ein. \\
 Daher wird mit einem neuen Modul f�r die in Abschnitt \ref{randbendingingen} beschriebene Applikation der gesamte Graph mit allen enthaltenen Knoten in eine statische \acs{HTML}-Struktur exportiert. 
 \subsection{Visualisierung des Graphen mit \acs{HTML}}
  Abbildung \ref{img:web} zeigt eine der durch die vorher beschriebe Methode entstandenen \acs{HTML}-Seiten. Die einzelnen Autoren untereinander enthalten \acs{HTML}-Verlinkungen zu allen Nachbarautoren und der Communities (pro jede Hierarchieebene eine) denen sie angeh�ren. 
 \begin{center}
					\hpic{\includegraphics[width=1\textwidth]{images/web}
					} \newcaption{HTML-Darstellung vom Autorknoten Paul Erd�s}
					\label{img:web}
	\end{center}
  Diese Art der Visualisierung verhindert, dass der Graph bei Betrachtung dynamisch aus der Datenbank geladen werden muss, da dies bereits zum Zeitpunkt des \acs{HTML}-Exports geschieht. Auf diese Art kann das Umfeld eines Autors effizient und schnell durchsucht werden. Durch einen alphabetischen Index k�nnen Autoren schnell gefunden und analysiert werden. 
% { Methoden }

% { Ergebnisse }

% { Diskussion }


\section{Fazit}
Die Ergebnisse aus Abschnitt \ref{communities} zeigen, dass sich im \acs{DBLP}-Datensatz die Autoren in Communities zusammenfinden. Diese Communities unterscheiden sich stark in Gr��e und in ihrem Einfluss (Conductance) auf andere Communities, wobei zwischen beiden Eigenschaften kein Zusammenhang erkennbar ist. Durch diese Communities entstehen neue M�glichkeiten die Autoren im Co-Author-Graphen zu klassifizieren. Autoren nehmen eine Rolle innerhalb ihrer Community an, wobei, bei der verwendeten Klassifizierung, nur verh�ltnism��ig wenige Autoren eine verbindende Rolle (Hub) einnehmen. \\


% { Ergebnisse }

% { Diskussion }

\section{Perspektiven}

Durch die modulare Architektur, der zur Messung erstellten Applikation, ist es einfach neue Messmodule in das System zu integrieren. Dadurch ist die Infrastruktur beliebig erweiterbar. Es k�nnten neue Messwerte wie beispielsweise die \textit{Significance} (Stabilit�t einer Community) eingef�hrt werden, um noch mehr Aussagen �ber die Autoren und deren Communities treffen zu k�nnen. Eine Betrachtung der zeitlichen Entwicklung von Communities und Autoren ist ebenfalls denkbar.\\
Weiterhin besteht die M�glichkeit durch die Entwicklung neuer Parser auch Datens�tze anderer Datensammlungen zu analysieren.



\section*{Abk�rzungsverzeichnis} 
\addcontentsline{toc}{section}{Abk�rzungsverzeichnis}

		\begin{acronym}
		 	\acro{DBLP}{DataBase systems and Logic Programming}
		 	\acro{XML}{Extensible Markup Language}
		 	
		\end{acronym}


\nocite{dblp_org}
\nocite{dblp_xml}
\nocite{conductance_formel}

\bibliographystyle{alphadin}
\bibliography{dblp_communities}

\end{document}
