
\section{Fazit}
Die Ergebnisse aus Abschnitt \ref{communities} zeigen, dass sich im \acs{DBLP}-Datensatz die Authoren in Communities zusammenfinden. Diese Communities unterscheiden sich stark in Gr��e und in ihrem Einfluss (Conductance) auf andere Communities, wobei zwischen beiden Eigenschaften kein Zusammenhang erkennbar ist. Durch diese Communities entstehen neue M�glichkeiten die Autoren im Co-Author-Graphen zu klassifizieren. Autoren nehmen eine Rolle innerhalb ihrer Community an, wobei, bei der verwendeten Klassifizierung, nur verh�ltnism��ig wenige Autoren eine verbindende Rolle (hub) einnehmen. \\


% { Ergebnisse }

% { Diskussion }

\subsection{Perspektiven}

Durch die modulare Architektur der zur Messung erstellten Applikation ist es einfach neue Messmodule in das System zu integrieren. Dadurch ist die Infrastruktur beliebig erweiterbar. Es k�nnten neue Messwerte wie beispielsweise die \textit{Significance} (Stabilit�t einer Community) eingef�hrt werden, um noch mehr Aussagen �ber die Autoren und deren Communities treffen zu k�nnen. Eine Betrachtung der zeitlichen Entwicklung von Communities und Autoren ist ebenfalls denkbar.\\
Weiterhin besteht die M�glichkeit durch die Entwicklung neuer Parser auch Datens�tze anderer Datensammlungen zu analysieren.

