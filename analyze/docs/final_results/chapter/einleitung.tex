\section{Einleitung}

Der \acs{DBLP} Datensatz beinhaltet eine gro�e Sammlung wissenschaftlicher Publikationen und deren Meta-Daten aus dem Bereich Informatik. Insgesamt sind bereits mehr als 1,8 Millionen Publikationen aufgenommen worden (\cite{dblp_org}). Aus diesen Daten l�sst sich ein Co-Autor-Graph ableiten, welcher alle Autoren als Knoten und Publikationen zwischen ihnen als gemeinsame Kanten beschreibt. Dieser Graph kann durch graphentheoretische Betrachtungen analysiert werden.\\ 

Anschaulich hei�t das: Haben zwei Autoren miteinander publiziert, sind deren Knoten durch eine Kante verbunden. Das Gewicht einer Kante zeigt zus�tzlich die Anzahl der gemeinsamen Publikationen an.\\

Bei einem Datensatz dieser Gr��e k�nnte eine Analyse dieses Graphen repr�sentative Aussagen �ber die Autoren und deren Umfeld geben. 
Somit wird die wissenschaftliche Fragestellung dieser Arbeit wie folgt festgelegt: \\\\

 Welche Charaktereigenschaften weisen die Autoren und ihr Umfeld im Co-Autorgraph der DBLP auf? \\\\

% { Methoden }
Um dieser Frage nachzukommen, wird eine Infrastruktur entwickelt, die es erm�glicht den Co-Autor-Graphen zu generieren und zu visualisieren. Zus�tzlich m�ssen Werkzeuge entwickelt werden, die es erlauben den Graphen zu analysieren. 

% {Motivation}
% {Hauptfrage: Welche Charaktereigenschaften weisen die Autoren und ihr Umfeld im Co-Autorgraph der DBLP auf?}

