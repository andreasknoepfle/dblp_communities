
\section{Communities im Co-Autor-Graphen}


Die Autoren sind durch gemeinsame Publikationen im Co-Autor-Graphen miteinander verbunden. Viele Publikationen zwischen zwei Autoren lassen auf eine starke Verbindung zwischen diesen schlie�en. Durch diese Verbindungen k�nnen sich Gemeinschaften unter den Autoren bilden.\\

% {Teilfrage: Lassen sich Gemeinschaften/Communities in den dblp Daten finden?}
Im Folgenden wird also untersucht, ob im Co-Autor-Graphen der DBLP Gemeinschaften unter den Autoren zu finden sind. \\
Sollten Gemeinschaften gefunden werden k�nnen, ergeben sich die weiteren Fragestellungen:

	\item Wie stark unterscheiden sich die Communities in ihrer Gr��e ?
	\item Lassen sich die Communities charakterisieren ?
	\item Ist ein Zusammenhang zwischen Gr��e und Charakter einer Community erkennbar ?


% { Methoden }
%Community-Algorithmus
Um diese Gemeinschaften, weiterhin mit dem englischen Begriff Communities bezeichnet, finden zu k�nnen, wird der in \cite{communityDetection} beschriebene Community-Detection Algorithmus eingesetzt. Dieser Algorithmus findet Communities und deren Zusammensetzung als Hierarchiestruktur. Die Ergebnisse des Community-Detection Algorithmus werden in der Neo4J Datenbank gespeichert. \\

Die oberste ebene der Hierarchiestruktur, die die Zusammensetzung der Communities beschreibt, besteht aus den gefundenen Communities (Top-Level-Communities). Jede dieser Communities besteht wieder aus kleineren Communities und diese wieder aus Communities usw.. So gibt es mehrere Stufen, wobei die unterste Ebene die Autoren darstellt. \\

%Gr��e der Community
Die Gr��e jeder Top-Level-Community und ihrer Untergruppen kann durch einen rekursiven Algorithmus berechnet werden. \\

%Conductance einer Community
Die Conductance einer Community zeigt das Verh�ltnis der Anzahl der Kanten, welche diese Community mit einer anderen verbindet, zu der Anzahl der Kanten innerhalb der Community an. \\

Mit diesen beiden Werten k�nnen die Communities charakterisiert werden.

% { Ergebnisse }




% { Diskussion }