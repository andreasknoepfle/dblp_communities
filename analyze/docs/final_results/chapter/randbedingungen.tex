
\section{Randbedingungen}
\label{randbendingingen}

Um den Co-Autor-Graphen zu analysieren, wird eine effiziente Form des Datenzugriffs ben�tigt. Um diesen Zugriff technisch umzusetzen wird im Weiteren die Graphdatenbank Neo4J verwendet. Mit Hilfe von Neo4J kann der Graph einfach und effizient gespeichert und traversiert werden. Zus�tzlich k�nnen Meta-Informationen zu den Knoten und Kanten in der Datenbank verwaltet werden, womit diese durch Analyseergebnisse angereichert werden k�nnen. Weiterhin unterst�tzt Neo4J die Entwicklung von Algorithmen auf Graphen und bietet bereits einige Implementierungen dieser.\\
Auf dieser Grundlage wird eine Appliaktion entwickelt, die sich aus vielen verscheidenen selbst�ndigen Modulen zusammensetzt. Alle Module sind nur �ber die Datenbasis Neo4j lose gekoppelt und k�nnen jeweils durch andere Module ersetzt werden. \\
Zum Zeitpunkt des Schreibens dieses Artikels sind folgende Module implementiert:
\begin{description}
  \item[parser] - Parsen der DBLP XML-Datei und erzeugung des Co-Author-Graphen und der Kantenliste f�r den Community-Algorithmus
  \item[importer] - Importieren der Ergebnisse des Community-Algorithmus
  \item[count] - Analyse der Gr��e der Communities
  \item[conductance] - Analyse der Conductance der Communitirs
  \item[roles] - Analyse der Rollen die die Autoren in ihrer Community �bernehmen
  \item[web] - Visualisieren der entstandenen Co-Author-Graphen
\end{description}
Die Entwicklung der Applikation ist �ffentlich einsehbar\footnote{\burl{https://github.com/andreasknoepfle/dblp_communities}}. 


%\subsubsection{Neo4J Datenbank}

%\subsubsection{Neo4J Datenbank}