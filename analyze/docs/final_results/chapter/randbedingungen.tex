
\section{Randbedingungen}

Um den Co-Autor-Graphen zu analysieren, wird eine effiziente Form des Datenzugriffs ben�tigt. Um diesen Zugriff technisch umzusetzen wird im Weiteren die Graphdatenbank Neo4J verwendet. Mit Hilfe von Neo4J kann der Graph einfach und effizient gespeichert und traversiert werden. Zus�tzlich k�nnen Meta-Informationen zu den Knoten und Kanten in der Datenbank verwaltet werden, womit diese durch Analyseergebnisse angereichert werden k�nnen. Weiterhin unterst�tzt Neo4J die Entwicklung von Algorithmen auf den Graphen und bietet bereits einige Implementierungen dieser.\\


%\subsubsection{Neo4J Datenbank}

%\subsubsection{Neo4J Datenbank}